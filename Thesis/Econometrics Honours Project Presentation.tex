%%%%%%%%%%%%%%%%%%%%%%%%%%%%%%%%%
%Preamble
%%%%%%%%%%%%%%%%%%%%%%%%%%%%%%%%%

\PassOptionsToPackage{usenames, dvipsnames, table}{xcolor}

\documentclass[]{beamer}
\usetheme{Boadilla}

%Make beamer number figures, because it doesn't by default
\setbeamertemplate{caption}[numbered]

\usepackage[english]{babel}
\usepackage[ruled, vlined]{algorithm2e}

\usepackage{amsfonts}
\usepackage{setspace, graphicx, epstopdf, amsmath}
%Do not use enumitem because this interferes with beamer drawing bullet points
\usepackage{marginnote, datetime, url, subfigure}

%Bibliography Stuff
%Use natbib even though it's old because it's compliant with journal styles
%Actual bibliography style etc are specified where you actually want it
\usepackage{natbib}


%Fluff
\linespread{1.3}

%Neural Network Packages
\usepackage{neuralnetwork}
\usepackage{xpatch}
\makeatletter
% \linklayers have \nn@lastnode instead of \lastnode,
% patch it to replace the former with the latter, and similar for thisnode
\xpatchcmd{\linklayers}{\nn@lastnode}{\lastnode}{}{}
\xpatchcmd{\linklayers}{\nn@thisnode}{\thisnode}{}{}
\makeatother

%Regression Tree
\usepackage{tikz,forest}
\usetikzlibrary{arrows.meta}

\forestset{
	.style={
		for tree={
			base=bottom,
			child anchor=north,
			align=center,
			s sep+=1cm,
			straight edge/.style={
				edge path={\noexpand\path[\forestoption{edge},thick,-{Latex}] 
					(!u.parent anchor) -- (.child anchor);}
			},
			if n children={0}
			{tier=word, draw, thick, rectangle}
			{draw, diamond, thick, aspect=2},
			if n=1{%
				edge path={\noexpand\path[\forestoption{edge},thick,-{Latex}] 
					(!u.parent anchor) -| (.child anchor) node[pos=.2, above] {Y};}
			}{
				edge path={\noexpand\path[\forestoption{edge},thick,-{Latex}] 
					(!u.parent anchor) -| (.child anchor) node[pos=.2, above] {N};}
			}
		}
	}
}

%%TODONOTE commands
\usepackage[colorinlistoftodos]{todonotes}
\newcommand{\smalltodo}[2][] {\todo[caption={#2}, size=\scriptsize,%
	fancyline,#1]{\begin{spacing}{.5}#2\end{spacing}}}
\newcommand{\rhs}[2][]{\smalltodo[color=green!30,#1]{{\bf RS:} #2}}
%%

%Graphs
\usepackage{tikz}
\usepackage{pgfplots}

%Coloured Tables


%%%%%%%%%%%%%%%%%%%%%%%%%%%%%%
%%Title and other fluff, just before document start
%%%%%%%%%%%%%%%%%%%%%%%%%%%%%%

%Hyperref apparently is a big package and causes a lot of issues, so it's recommended to load this last

\usepackage{hyperref}

%Gets rid of the neon green boxes around boxes

\usepackage[]{xcolor}

\hypersetup{
	colorlinks,
	linkcolor = {red!50!black},
	citecolor = {blue!50!black},
	urlcolor = {blue!80!black}
}

\title{Evaluation of Machine Learning in Finance}

\author{
	Ze Yu Zhong \\
	Supervisor: David Frazier
}

\institute{Monash University}

\date{}

\begin{document}
	
\begin{frame}[plain]
    \maketitle
\end{frame}

%%%%%%%%%%%%%%%%%%%%%%%%%%%%%%%%%%%%%%%%%%%%%%%%%%%%
\section{Background}
%%%%%%%%%%%%%%%%%%%%%%%%%%%%%%%%%%%%%%%%%%%%%%%%%%%%

%%%%%%%%%%%%%%%%%%%%%%%%%%%%%%%%%%%%%%%%%%%%%%%%%%%%%
%%Intereave literature review throughout this section
%%%%%%%%%%%%%%%%%%%%%%%%%%%%%%%%%%%%%%%%%%%%%%%%%%%%%

\begin{frame}
\frametitle{Main Motivation}
To evaluate the application of machine learning to predicting financial asset returns, with specific regard to how they deal with the unique challenges present in financial data via simulation.
\end{frame}

\begin{frame}
\frametitle{Background}
\begin{itemize}
\item Factors: a collection of regressors to be used in pricing returns that can be used to proxy for unknown underlying risk factors due to their correlation with cross sectional returns, \citep{harvey__2016}

	\begin{itemize}
		\item Violates strict view that risk factors should be variables that have unpredictable variation through time, and be able to explain cross sectional returns independently
	\end{itemize}
\end{itemize}

\end{frame}

\begin{frame}
\frametitle{Background - Dividend Ratio Example}
\begin{itemize}

\item Included due to good in sample performance in the 1990s \citep{goyal_predicting_2003}

\item \textit{Persistent} (\cite{goetzmann_testing_1993}, \cite{ang_stock_2006})

	\begin{itemize}
	\item Correlated with lagged dependent variables on the right hand side of the regression equation. 
	
	\item Violates assumptions of independent regressors of OLS: t stats are biased upwards due to autocorrelated errors
	
	\item GMM and NW errors corrections are also biased, \citep{goetzmann_testing_1993}
	\end{itemize}

\item Not robust and have poor out of sample performance since 2000s (\cite{goyal_predicting_2003}, \cite{lettau_consumption_2001}, \cite{schwert_anomalies_2003})
\end{itemize}
\end{frame}

\begin{frame}
\frametitle{Dividend Ratio Example}
\begin{itemize}
\item Factors such as dividend ratios, earnings price ratio, interest and inflation etc. were ``widely accepted" able to predict excess returns, \citep{lettau_consumption_2001}

\item \cite{welch_comprehensive_2008} conclude that not a single variable had any statistical forecasting power, and the significance values of some factors change with the choice of sample periods.
\end{itemize}
\end{frame}

\begin{frame}
\frametitle{Background}
\begin{itemize}
\item More factors produced by literature: currently over 600 documented \citep{harvey_census_2019}

\begin{itemize}
	\item False discovery problem, \citep{harvey__2016}
	
	\item Factors are cross sectionally correlated - inefficient covariances, factors may be subsumed within others, \citep{feng_taming_2019}
	
	\item Number of factors may be more than sample size, making regression impossible
\end{itemize}


\end{itemize}
\end{frame}

%%%%%%%%%%%%%%%%%%%%%%%%%%%%%%%%%%%%%%%%%%%%%%%%%%%%
\section{What is Machine Learning?}
%%%%%%%%%%%%%%%%%%%%%%%%%%%%%%%%%%%%%%%%%%%%%%%%%%%%

\begin{frame}
\frametitle{What is Machine Learning?}
``...a vast set of tools for understanding data."

- \textit{An Introduction to Statistical Learning}, \citep{hastie_elements_2009}

We will define it as a diverse collection of:
\begin{itemize}
	\item high dimensional models for statistical prediction,
	\item ``regularization" methods for model selection and mitigation of overfitting in sample data
	\item efficient systematic methods for searching potential model specifications
\end{itemize}
\end{frame}

%%%%%%%%%%%%%%%%%%%%%%%%%%%%%%%%%%%%%%%%%%%%%%%%%%%%
\section{Why apply Machine Learning in Finance?}
%%%%%%%%%%%%%%%%%%%%%%%%%%%%%%%%%%%%%%%%%%%%%%%%%%%%

%%%%%%%%%%%%%%%%%
%%Interweave Literature review throughout this section
%%%%%%%%%%%%%%%%%

%\begin{frame}
%\frametitle{Why apply Machine Learning in Finance?}
%\begin{itemize}
%	\item High dimensional - more flexible than traditional regression models, which make strong functional form assumptions and are sensitive to outliers, \citep{freyberger_dissecting_2017}
%	\item Explicit methods for guarding against overfitting and generalizing poorly
%	\item Methods to produce an optimal model from all possible at manageable computation cost
%\end{itemize}
%\end{frame}

\begin{frame}
\frametitle{Applications in the Literature}
\begin{itemize}
	\item \cite{kozak_shrinking_2017}, \cite{rapach_forecasting_2013}, \cite{freyberger_dissecting_2017}, and others apply shrinkage and selection methods to identify important factors
	\item \cite{gu_empirical_2018}, \cite{feng_deep_2018}, construct machine learning portfolios that historically outperform traditional portfolios in terms of prediction error and predictive $R^2$
	\item Attribute their success to machine learning's ability to find non-linear interactions
\end{itemize}
\end{frame}

\begin{frame}
\frametitle{Motivations}
However, little work has been done on how machine learning actually recognises and deals with the challenges in financial data. 
\begin{itemize}
	\item \cite{feng_deep_2018} cross validates their training set, destroying temporal aspect of data, and only explore a handful of factors
	\item \cite{gu_empirical_2018} only use data up until the 1970s to produce predictions in the last 30 years
	\item \cite{gu_empirical_2018}'s models do not have consistent importance metrics - only their tree based methods recognise dividend yield as important
\end{itemize}
\end{frame}

\begin{frame}
\frametitle{Motivations}
\begin{itemize}
\item Can machine learning deal with the challenges in financial data?
\begin{itemize}
	\item Persistent Regressors?
	\item Identify true factors from a high dimensional, cross sectionally correlated panel?
	\item Is regularization enough to handle non-robustness?
	\item Are their conclusions consistent?
	\item Do they perform better than traditional methods?
\end{itemize}

\item Explore this via simulation and popular machine learning models
\item Models will also be evaluated again, but with more recent, representative financial data to explore robustness. 
\end{itemize}
\end{frame}

%%%%%%%%%%%%%%%%%%%%%%%%%%%%%%%%%%%%%%%%%%%%%%%%%%%%
\section{Model Specification}
%%%%%%%%%%%%%%%%%%%%%%%%%%%%%%%%%%%%%%%%%%%%%%%%%%%%

\begin{frame}
\frametitle{Model Overview}
Returns are modelled as an additive error model
\begin{equation}
	r_{i, t+1} = E(r_{i, t+1} | \mathcal{F}_t) + \epsilon_{i, t+1}
\end{equation}
		
where 
\begin{equation}
	E(r_{i, t+1} | \mathcal{F}_t) = g^*(z_{i,t})
\end{equation}
		
Stocks are indexed as $i = 1, \dots, N$ and months by $t = 1, \dots, T$. 

$g^*(z_{i,t})$ represents the model approximation using the $P$ dimensional predictor set $z_{i,t}$. 
\end{frame}

%%%%%%%%%%%%%%%%%%%%%%%%%%%%%%%%%%%%%%%%%%%%%%%%%%%%
\section{Methodology}
%%%%%%%%%%%%%%%%%%%%%%%%%%%%%%%%%%%%%%%%%%%%%%%%%%%%

\begin{frame}
\frametitle{Overview}
Machine Learning Methodology consists of 3 overall components:
\begin{itemize}
	\item Sample Splitting
	\item Loss Functions
	\item Models/Algorithms considered
\end{itemize}
\end{frame}

%%%%%%%%%%%%%%%%%%%%%%%%%%%%%%%%%%%%%%%%%%
%Sample Splitting
%%%%%%%%%%%%%%%%%%%%%%%%%%%%%%%%%%%%%%%%%%

\begin{frame}
\frametitle{Expanding/Growing Window Approach}
An expanding/growing window approach was used to
\begin{itemize}
	\item Allow models to incorporate more data over time
	\item Preserve temporal ordering of data (compared to cross-validation)
	\item Allows the model to use the most recent data, in some sense
\end{itemize}
The training/validation split was chosen such that the size of the training set was 1.5 times the length of the validation set to begin with, consistent with \cite{gu_empirical_2018}.
Split specification is ultimately subjective
\end{frame}

\begin{frame}
\frametitle{Sample Splitting}
\begin{figure}
	\begin{center}
		\begin{tabular}{|c|p{0.25cm}p{0.25cm}p{0.25cm}p{0.25cm}p{0.25cm}p{0.25cm}p{0.25cm}p{0.25cm}p{0.25cm}p{0.25cm}p{0.25cm}p{0.25cm}|p{0.25cm}p{0.25cm}p{0.25cm}|}
			\hline
			Set No. &&&&&&&&&&&&&&& \\
			\hline
			%%%%%%%%
			3 & \cellcolor{cyan} & \cellcolor{cyan} & \cellcolor{cyan} & \cellcolor{cyan} & \cellcolor{cyan} & \cellcolor{cyan} & \cellcolor{cyan} & \cellcolor{cyan} & \cellcolor{cyan} & \cellcolor{cyan} & \cellcolor{cyan} &
			\cellcolor{pink} & 
			\cellcolor{olive} & \cellcolor{olive} &	\cellcolor{olive} \\
			%%%%%%%%
			2 & \cellcolor{cyan} & \cellcolor{cyan} & \cellcolor{cyan} & \cellcolor{cyan} & \cellcolor{cyan} & \cellcolor{cyan} & \cellcolor{cyan} & \cellcolor{cyan} & \cellcolor{cyan} & \cellcolor{cyan} &
			\cellcolor{pink} & 
			\cellcolor{olive} & \cellcolor{olive} &	\cellcolor{olive} & \cellcolor{olive} \\
			%%%%%%%%
			1 & \cellcolor{cyan} & \cellcolor{cyan} & \cellcolor{cyan} & \cellcolor{cyan} & \cellcolor{cyan} & \cellcolor{cyan} & \cellcolor{cyan} & \cellcolor{cyan} & \cellcolor{cyan} &
			\cellcolor{pink} & 
			\cellcolor{olive} & \cellcolor{olive} &	\cellcolor{olive} & \cellcolor{olive} & \cellcolor{olive} \\
			\hline
			Year & 1 & 2 & 3 & 4 & 5 & 6 & 7 & 8 & 9 & 10 & 11 & 12 & 13 & 14 & 15\\
			\hline
		\end{tabular}
		\medskip
		\begin{tabular}{|c|p{0.25cm}|}
			\hline
			Training & \cellcolor{cyan} \\
			\hline
			Validation & \cellcolor{pink} \\
			\hline
			Test & \cellcolor{olive} \\
			\hline
		\end{tabular}
	\end{center}
	\caption{Sample Splitting Procedure}
\end{figure}
\end{frame}

%%%%%%%%%%%%%%%%%%%%%%%%%%%%%%%%%%%%%%%%%%
%Loss Functions
%%%%%%%%%%%%%%%%%%%%%%%%%%%%%%%%%%%%%%%%%%

\begin{frame}
\frametitle{Loss Functions}
Mean Absolute Error (MAE)

	\begin{equation}
	\text{MAE} = \frac{1}{n} \sum_{j = i}^{n} |y_j - \hat{y_j}|
	\end{equation}
	
Mean Squared Error (MSE)

	\begin{align}
	\text{MSE} &= \frac{1}{n} \sum_{j = i}^{n} \left( y_j - \hat{y_j}\right) ^2
	\end{align}
	
% Huber Loss
%		\begin{align}
%		H(\epsilon_j = y_j - \hat{y_j};\xi) = 
%		\begin{cases}
%		\left( y_j - \hat{y_j}\right) ^2, 
%		\quad &\text{if} \quad |y_j - \hat{y_j}| \leq \xi ; \\
%		2 \xi  |y_j - \hat{y_j}| - \xi^2, 
%		\quad &\text{if} \quad |y_j - \hat{y_j}| > \xi
%		\end{cases}
%		\end{align}

\end{frame}

%\begin{frame}
%\frametitle{Loss Functions}
%\begin{figure}
%	\begin{center}
%		\begin{tikzpicture}
%		\begin{axis}[ xlabel={$\epsilon$}, ylabel={Loss}, axis lines=middle, samples=41, grid, thick, domain=-2:2]
%		\addplot+[no marks] {abs(x)};
%		\addlegendentry{MAE}
%		\addplot+[no marks] {x^2};
%		\addlegendentry{MSE}
%		\addplot+[no marks] {(2 * abs(x) - 1)*(abs(x) > 1) +
%			(x^2)*(abs(x) <= 1 )};
%		\addlegendentry{Huber Loss}
%		\end{axis}
%		\end{tikzpicture}
%	\end{center}
%	\caption{Illustration of MAE, MSE and Huber Loss when $\xi = 1$}
%	\label{fig:loss_functions}
%\end{figure}
%\end{frame}

%%%%%%%%%%%%%%%%%%%%%%%%%%%%%%%%%%%%%%%%%%
%Linear Models
%%%%%%%%%%%%%%%%%%%%%%%%%%%%%%%%%%%%%%%%%%

\begin{frame}
\frametitle{Linear Models}
Linear Models assume that the underlying conditional expectation \( g^*(z_{i, t}) \) can be modelled as a linear function of the predictors and the parameter vector \( \theta \):
	\begin{equation}
	g(z_{i, t};\theta) = z_{i, t}' \theta
	\end{equation}
	
Optimizing $\theta$ w.r.t. MSE yields the Pooled OLS estimator
\end{frame}

%%%%%%%%%%%%%%%%%%%%%%%%%%%%%%%%%%%%%%%%%%
%Penalized Linear
%%%%%%%%%%%%%%%%%%%%%%%%%%%%%%%%%%%%%%%%%%

\begin{frame}
\frametitle{Penalized Linear Models}
Linear Models + Penalty term (Elastic Net by \cite{zou_regularization_2005} shown):
	\begin{align}
	\mathcal{L(\theta;.)} &= 
		\underset{\text{Loss Function}}{\underbrace{\mathcal{L(\theta)}}} + 
		\underset{\text{Penalty Term}}{\underbrace{\phi(\theta;.)}} \\
	\phi(\theta;\lambda,\rho) &= 
		\lambda(1-\rho) \sum_{j = 1}^{P}|\theta_j| +
		\frac{1}{2} \lambda \rho \sum_{j = 1}^{P}\theta_j^2
	\end{align}

Elastic Net penalty produces efficient and parsimonious via shrinkage and selection
	
\end{frame}

%%%%%%%%%%%%%%%%%%%%%%%%%%%%%%%%%%%%%%%%%%
%Regression Trees and Random Forests
%%%%%%%%%%%%%%%%%%%%%%%%%%%%%%%%%%%%%%%%%%

\begin{frame}
\frametitle{Regression Trees \& Random Forests}
\begin{itemize}
	\item Fully non-parametric models that can capture complex multi-way interactions. 
	\item A tree "grows" in a series of iterations:
	\begin{enumerate}
		\item Make a split ("branch") along one predictor, such that it is the best split available at that stage with respect to minimizing the loss function
		\item Repeat until each observation is its own node, or until the stopping criterion is met
	\end{enumerate}
	\item Slices the predictor space into rectangular partitions, and predicts the unknown function $g^*(z_{i,t})$ with the ``average" value of the outcome variable in each partition to minimize the loss function
\end{itemize}
\end{frame}

\begin{frame}
\frametitle{Random Forests}
Trees have very low bias and high variance

They are very prone to overfitting and non-robust

Random Forests were proposed by \cite{breiman_random_2001} to address this
\begin{itemize}
	\item Create $B$ bootstrap samples
	\item Grow a highly overfit tree to each, but only using $m$ random subset of all predictors for each
	\item Average the output from all trees as an ensemble model
\end{itemize}
\end{frame}

%%%%%%%%%%%%%%%%%%%%%%%%%%%%%%%%%%%%%%%%%%%%%%%%%
%Neural Networks
%%%%%%%%%%%%%%%%%%%%%%%%%%%%%%%%%%%%%%%%%%%%%%%%%

\begin{frame}
\frametitle{Neural Networks}
\begin{align}
	x_k^{(l)} &= \alpha(x^{(l-1)'}\theta_k^{l-1}) \\
	x_1^{(1)} &= \alpha \left( (x_0^{(0)}, x_1^{(0)}, x_2^{(0)}, x_3^{(0)})'\theta_1^{0} \right) 
\end{align}

\begin{figure}
	\begin{neuralnetwork}
		%Options
		[nodespacing=9.5mm, layerspacing=18mm,
		maintitleheight=1em, layertitleheight=2em,
		height=7, toprow=false, nodesize=24pt, style={},
		title={}, titlestyle={}]
		\newcommand{\nodetextclear}[2]{}
		%use \ifnum to get different labels, such as x_n on the last neuron
		\newcommand{\nodetextx}[2]{\ifnum ####2=8 $x_n^{(0)}$ \else $x_####2^{(0)}$ \fi}
		\newcommand{\nodetexty}[2]{$y_####2$}
		%Hidden layer textcommands
		%32 neurons
		\newcommand{\nodetextxa}[2]{\ifnum ####2=7 $x_{32}^{(1)}$ \else $x_####2^{(1)}$ \fi}
		%16 neurons
		\newcommand{\nodetextxb}[2]{\ifnum ####2=6 $x_{16}^{(2)}$ \else $x_####2^{(2)}$ \fi}
		%8 neurons
		\newcommand{\nodetextxc}[2]{\ifnum ####2=5 $x_{8}^{(3)}$ \else $x_####2^{(3)}$ \fi}
		\newcommand{\nodetextxd}[2]{$x_####2^{(4)}$}
		\newcommand{\nodetextxe}[2]{$x_####2^{(5)}$}
		%Input Layer
		\inputlayer[count=3, bias=true, title=, text=\nodetextx]
		%Hidden Layer 1
		\hiddenlayer[count=1, bias=true, title=, text=\nodetextxa] 
		\linklayers[]
		%Final Layer
		\outputlayer[count=1, title=, text=\nodetexty] \linklayers
	\end{neuralnetwork}
\caption{Sample Neural Network}
\end{figure}
\end{frame}

\begin{frame}
\frametitle{Neural Network Specifications}

\begin{itemize}
\item Neural networks with up to 5 hidden layers were considered. 

\item The number of neurons is each layer determined by geometric pyramid rule \citep{masters_practical_1993}

\item All units are fully connected
\end{itemize}

ReLU activation function was chosen for all hidden layers for computational speed, and hence popularity in literature:

\begin{equation}
\operatorname{ReLU}(x) = max(0, x)
\end{equation}
\end{frame}

%\begin{frame}
%\frametitle{Computation}
%\begin{itemize}
%	\item Stochastic Gradient Descent using ADAM
%	\item Batch Normalization
%	\item Randomize initial starting weights and biases, then average these into an ensemble model
%	\item See references for specific details
%\end{itemize}
%\end{frame}

\begin{frame}
%Beamer hack snad uses # in a loop
%Therefore, we need to replace all # with ####
\begin{figure}
	\begin{neuralnetwork}
		%Options
		[nodespacing=9.5mm, layerspacing=18mm,
		maintitleheight=1em, layertitleheight=2em,
		height=7, toprow=false, nodesize=20pt, style={},
		title={}, titlestyle={}]
		\newcommand{\nodetextclear}[2]{}
		%use \ifnum to get different labels, such as x_n on the last neuron
		\newcommand{\nodetextx}[2]{\ifnum ####2=8 $x_n^{(0)}$ \else $x_####2^{(0)}$ \fi}
		\newcommand{\nodetexty}[2]{$y_####2$}
		%Hidden layer textcommands
		%32 neurons
		\newcommand{\nodetextxa}[2]{\ifnum ####2=7 $x_{32}^{(1)}$ \else $x_####2^{(1)}$ \fi}
		%16 neurons
		\newcommand{\nodetextxb}[2]{\ifnum ####2=6 $x_{16}^{(2)}$ \else $x_####2^{(2)}$ \fi}
		%8 neurons
		\newcommand{\nodetextxc}[2]{\ifnum ####2=5 $x_{8}^{(3)}$ \else $x_####2^{(3)}$ \fi}
		\newcommand{\nodetextxd}[2]{$x_####2^{(4)}$}
		\newcommand{\nodetextxe}[2]{$x_####2^{(5)}$}
		%Input Layer
		\inputlayer[count=8, bias=false, exclude = {7}, title=, text=\nodetextx]
		%Hidden Layer 1
		\hiddenlayer[count=7, bias=false, exclude = {6}, title=, text=\nodetextxa] 
		\linklayers[not from = {7}, not to = {6}]
		%Hidden Layer 2
		\hiddenlayer[count=6, bias=false, exclude = {5}, title=, text=\nodetextxb] 
		\linklayers[not from = {6}, not to = {5}]
		%Hidden Layer 3
		\hiddenlayer[count=5, bias=false, exclude = {4}, title=, text=\nodetextxc] 
		\linklayers[not from = {5}, not to = {4}]
		%Hidden Layer 4
		\hiddenlayer[count=4, bias=false, title=, text=\nodetextxd] 
		\linklayers[not from = {4}]
		%Hidden Layer 5
		\hiddenlayer[count=2, bias=false, title=, text=\nodetextxe] \linklayers
		%Final Layer
		\outputlayer[count=1, title=, text=\nodetexty] \linklayers
		% draw dots
		\path (L0-6) -- node{$\vdots$} (L0-8);
		\path (L1-5) -- node{$\vdots$} (L1-7);
		\path (L2-4) -- node{$\vdots$} (L2-6);
		\path (L3-3) -- node{$\vdots$} (L3-5);
	\end{neuralnetwork}
	\caption{Neural Network 5 (most complex considered)}
	\label{Neural_Network}
\end{figure}
\end{frame}

%%%%%%%%%%%%%%%%%%%%%%%%%%%%%%%%%%%%%%%%%%%%%%%%%%%%
\section{Simulation}
%%%%%%%%%%%%%%%%%%%%%%%%%%%%%%%%%%%%%%%%%%%%%%%%%%%%

\subsection{Simulation Design}
\begin{frame}
\frametitle{Overall Simulation Design}
Simulate a latent factor model with stochastic volatility for excess return, $r_{t+1}$, for $t=1,\dots,T$:

\begin{flalign}
r_{i, t+1} &= 
g\left(z_{i, t}\right) + \beta_{i,t+1}v_{t+1} + e_{i, t+1}; \\
z_{i, t} &= \left(1, x_{t}\right)^{\prime} \otimes c_{i, t}; 
\quad \beta_{i, t}=\left(c_{i 1, t}, c_{i 2, t}, c_{i 3, t}\right); \\ 
e_{i, t+1} &= 
\exp\left( \frac{\sigma_{i, t+1}^2}{2} \right) \varepsilon_{i, t+1}; \\
\sigma^2_{i,t+1} &= 
\omega + \gamma_i\sigma^2_{t,i}+w_{i,t+1}
\end{flalign}

$v_{t+1}$ is a $3\times 1$ vector of errors, $w_{i,t+1},\varepsilon_{i,t+1}$ are scalar error terms. Variances tuned such that the R squared for each individual return series was 50\% and annualized volatility 30\%.

\end{frame}

\begin{frame}
\frametitle{Simulating Characteristics}

Matrix $C_t$ is an $N\times P_c$ vector of latent factors. 

$x_t$ is a $3 \times 1$ multivariate time series

$\varepsilon_{t+1}$ is a $N\times 1$ vector of idiosyncratic errors. 

Simulation mechanism for $C_t$ that gives correlation across the factors \& time

Draw normal random numbers for each $1\leq i\leq N$ and $1\leq j\leq P_{c}$, according to 

\begin{equation}
\overline{c}_{i j, t} = \rho_{j} \overline{c}_{i j, t-1}+\epsilon_{i j, t} ;
\quad \rho_{j} \sim \mathcal{U} \left( \frac{1}{2},1 \right) 
\end{equation}

\end{frame}

\begin{frame}
\frametitle{Simulating Characteristics}
Then, define the matrix 

\begin{equation}
B:=\Lambda\Lambda' + \frac{1}{10}\mathbb{I}_{n}, \quad
\Lambda_i = (\lambda_{i1},\dots,\lambda_{i4}), \quad
\lambda_{ik}\sim N(0,1), \; k=1,\dots,4
\end{equation}

Transform this into a correlation matrix $W$ via

\begin{equation}
W = \left( \operatorname{diag}(B) \right) ^{\frac{-1}{2}}
(B)
\left( \operatorname{diag}(B) \right) ^{\frac{-1}{2}}
\end{equation}

Use $W$ to build in cross sectional correlation for $N\times P_{c}$ matrix $\bar{C}_t$:

\begin{equation}
\widehat{C}_{t}=W\overline{C}_{t}
\end{equation}
\end{frame}

\begin{frame}
\frametitle{Simulating Characteristics}

Finally, the "observed" characteristics for each $1\leq i\leq N$ and for $j=1, \dots, P_{c}$ are constructed according to:

\begin{equation}
c_{i j, t} = \frac{2}{n+1} \operatorname{rank}\left(\hat{c}_{i j, t}\right) - 1.
\end{equation}

with the rank transformation normalizing all predictors to be within $[-1, 1]$ 
\end{frame}

\begin{frame}
\frametitle{Simulating Macroeconomic Time Series}
For simulation of $x_{t}$, a $3 \times 1$ multivariate time series, we consider a VAR model:
\begin{equation*}
x_{t}=Ax_{t-1}+u_t, 
\quad u_t \sim N\left( \mu = (0, 0, 0)' , \Sigma = \mathbb{I}_{3}
\end{equation*}

\[
A_1 =
\begin{bmatrix}
.95 & 0 & 0 \\
0 & .95 & 0 \\
0 & 0 & .95
\end{bmatrix} ;
A_2 =
\begin{bmatrix}
1 & 0 & .25 \\
0 & .95 & 0 \\
.25 & 0 &.95
\end{bmatrix} ;
A_3 =
\begin{bmatrix}
.99 & .20 & .10 \\
.20 & .90 & -.30 \\
.10 & -.30 & -.99
\end{bmatrix}
\]
\end{frame}

\begin{frame}
\frametitle{Simulating Return Series}
We will consider four different functions $g(\cdot)$:
\begin{flalign*}
(1)\; & g_1 \left(z_{i, t}\right)=\left(c_{i 1, t}, c_{i 2, t}, c_{i 3, t} \times x_{3,t}'\right) \theta_{0} \\
(2)\; & g_2 \left(z_{i, t}\right)=\left(c_{i 1, t}^{2}, c_{i 1, t} \times c_{i 2, t}, \operatorname{sgn}\left(c_{i 3, t} \times  x_{3,t}'\right)\right) \theta_{0} \\
(3)\; & g_3 \left(z_{i, t}\right) = \left(1[c_{i3,t}>0],c_{i 2, t}^{3}, c_{i 1, t} \times c_{i 2, t}\times 1[c_{i3,t}>0], \text{logit}\left({c}_{i 3, t} \right)\right) \theta_{0} \\
(4)\; & g_4 \left(z_{i, t}\right)=\left(\hat{c}_{i 1, t}, \hat{c}_{i 2, t}, \hat{c}_{i 3, t} \times x_{3,t}'\right) \theta_{0}
\end{flalign*}

Tune $\theta^0$ s.t. cross sectional $R^2$ is 25\%, and predictive $R^2$ is 5\%. 

The simulation design results in $3 \times 4 = 12$ different simulation designs, with $N = 200$ stocks, $T = 180$ periods and $P_c = 100$ characteristics. Each design will be simulated 50 times to assess the robustness of machine learning algorithms.
\end{frame}

%%%%%%%%%%%%%%%%%%%%%%%%%%%%%%%%%%%%%%%%%%%%%%%%%%%%
\section{Real Data}
%%%%%%%%%%%%%%%%%%%%%%%%%%%%%%%%%%%%%%%%%%%%%%%%%%%%

\begin{frame}
\frametitle{Data Source}
\begin{itemize}
\item CRSP/Compustat database for stock returns with stock level characteristics such as accounting ratios and macroeconomic factors will be queried.

\item Only more recent data will be used, such as the period before and after 2008 GFC
\end{itemize}

\end{frame}

%%%%%%%%%%%%%%%%%%%%%%%%%%%%%%%%%%%%%%%%%%%%%%%%%%%%
\section{Model Evaluation}
%%%%%%%%%%%%%%%%%%%%%%%%%%%%%%%%%%%%%%%%%%%%%%%%%%%%

\begin{frame}
\frametitle{Out of Sample R Squared}
Overall predictive performance for individual excess stock returns were assessed using the out of sample $R^2$:

\begin{equation}
R^2_{OOS} = 
	1 - 
	\frac{\sum_{(i, t)\in\mathcal{T}_3}(r_{i, t+1} - \widehat{r}_{i, t+1})}
	{\sum_{(i, t)\in\mathcal{T}_3} \left( r_{i, t+1} - \bar{r}_{i, t+1} \right) ^2}
\end{equation}

where $\mathcal{T}_3$ indicates that the fits are only assessed on the test subsample
\end{frame}

\begin{frame}
\frametitle{Diebold Mariano Tests for Predictive Accuracy}
\begin{itemize}
	\item Compares the forecast accuracy of two forecast methods, (\cite{diebold_comparing_2002} and \cite{harvey_testing_1997})
	
	\item Tests whether or not the difference series ($d_t = e_{1t} - e_{2t}$) between two forecast methods' errors is different from zero
	
	\item $e_{1t}$ and $e_{2t}$ represent the average forecast errors for each model
\end{itemize}
\end{frame}

%\begin{frame}
%\frametitle{Diebold Mariano Tests for Predictive Accuracy}
%Under the null hypothesis:
%
%\begin{align}
%S_1^* &= \left[ 
%\frac{n + 1 - 2h + n^{-1}h(h-1)}
%{n} 
%\right]^{1/2}S_1 ; \quad S_1^* \sim N(0,1)\\
%S_1 &= \left[ 
%\hat{V}(\bar{d})
%\right] ^{-1/2}\bar{d} \\
%\hat{\gamma}_k &= n^{-1} \sum_{t = k + 1}^{n}(d_t - \bar{d})(d_{t-k} - \bar{d}) \\
%V(\bar{d}) &\approx n^{-1}\left[ 
%\gamma_0 + 2 \sum_{k = 1}^{h - 1}\gamma_k
%\right] 
%\end{align}
%
%where $d_t$ = $e_{1t} - e_{2t}$.
%\end{frame}

\begin{frame}
\frametitle{Variable Importance}

\begin{itemize}
\item The importance of each predictor $j$ is denoted as $VI_j$

\item Defined as the reduction in predictive R-Squared from setting all values of predictor $j$ to 0, while holding the remaining model estimates fixed

\item Will allow us to see what factors the models have determined to be important
\end{itemize}

\end{frame}

%%%%%%%%%%%%%%%%%%%%%%%%%%%%%%%%%%%%%%%%%%%%%%%%%%%%
\section{Results}
%%%%%%%%%%%%%%%%%%%%%%%%%%%%%%%%%%%%%%%%%%%%%%%%%%%%

\begin{frame}
\frametitle{Results}
Work is currently being done on trying to tune the R-Squared values of the simulated datasets
\end{frame}

%%%%%%%%%%%%%%%%%%%%%%%%%%%%%%%%%%%%%%%%%%%%%%%%%%%%
\section{References}
%%%%%%%%%%%%%%%%%%%%%%%%%%%%%%%%%%%%%%%%%%%%%%%%%%%%
%This is just here to get the citations working throughout the presentation, skip over this when presenting
\begin{frame}
\frametitle{References}
\bibliographystyle{jfe}
\bibliography{Bibliography}
\end{frame}

%%%%%%%%%%%%%%%%%%%%%%%%%%%%%%%%%%%%%%%%%%%%%%%%%%%%
\section{Questions and Answers}
%%%%%%%%%%%%%%%%%%%%%%%%%%%%%%%%%%%%%%%%%%%%%%%%%%%%

\begin{frame}
\begin{center}
\huge Questions and Answers
\end{center}
\end{frame}

\end{document}
