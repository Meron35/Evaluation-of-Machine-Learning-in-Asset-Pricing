%%%%%%%%%%%%%%%%%%%%%%%%%%%%%%%%%
%Preamble
%%%%%%%%%%%%%%%%%%%%%%%%%%%%%%%%%

\documentclass[a4paper]{article}
\usepackage[english]{babel}
\usepackage[margin=1in]{geometry}
\usepackage[ruled, vlined]{algorithm2e}

\usepackage{amsfonts}
\usepackage{setspace,graphicx,epstopdf,amsmath}
\usepackage{marginnote, datetime, url, enumitem, subfigure}

%Journal Style
%JFE looks nice, JF looks awful
\usepackage{amsthm}
\usepackage{jfe}

%Bibliography Stuff
%Use natbib even though it's old because it's compliant with journal styles
%Actual bibliography style etc are specified where you actually want it
\usepackage{natbib}

%Fluff
\linespread{1.3}

%Neural Network Packages
\usepackage{neuralnetwork}
\usepackage{xpatch}
\makeatletter
% \linklayers have \nn@lastnode instead of \lastnode,
% patch it to replace the former with the latter, and similar for thisnode
\xpatchcmd{\linklayers}{\nn@lastnode}{\lastnode}{}{}
\xpatchcmd{\linklayers}{\nn@thisnode}{\thisnode}{}{}
\makeatother

%Regression Tree
\usepackage{tikz,forest}
\usetikzlibrary{arrows.meta}

\forestset{
	.style={
		for tree={
			base=bottom,
			child anchor=north,
			align=center,
			s sep+=1cm,
			straight edge/.style={
				edge path={\noexpand\path[\forestoption{edge},thick,-{Latex}] 
					(!u.parent anchor) -- (.child anchor);}
			},
			if n children={0}
			{tier=word, draw, thick, rectangle}
			{draw, diamond, thick, aspect=2},
			if n=1{%
				edge path={\noexpand\path[\forestoption{edge},thick,-{Latex}] 
					(!u.parent anchor) -| (.child anchor) node[pos=.2, above] {Y};}
			}{
				edge path={\noexpand\path[\forestoption{edge},thick,-{Latex}] 
					(!u.parent anchor) -| (.child anchor) node[pos=.2, above] {N};}
			}
		}
	}
}

%%TODONOTE commands
\usepackage[colorinlistoftodos]{todonotes}
\newcommand{\smalltodo}[2][] {\todo[caption={#2}, size=\scriptsize,%
	fancyline,#1]{\begin{spacing}{.5}#2\end{spacing}}}
\newcommand{\rhs}[2][]{\smalltodo[color=green!30,#1]{{\bf RS:} #2}}
%%

%Graphs
\usepackage{pgfplots}

%opening
\title{Evaluation of Machine Learning in Empirical Asset Pricing Proposal}
\author{Ze Yu Zhong}

\begin{document}

\maketitle

\todo{Remember that the entire document is meant to be no more than 5 pages, 10 max, not including appendix}

\section{Statement of Topic}

This thesis aims to evaluate the application of machine learning algorithms in empirical asset pricing. While there has been 

\section{Background Material}

\todo{includes motivation and brief review of key literature}

\section{Significance of Proposed Study}

This study will be the first to 

In addition, 

\section{Research Plan}

We first evaluate machine learning algorithms on a collection of simulated datasets which include cross sectionally correlated characteristics which enter the return equation in a range of linear and highly non-linear combinations.

\subsection{Models}

The thesis will focus only on four different models, chosen for their prevalence and popularity in the literature.

\subsubsection{Linear Model}

\subsubsection{Penalized Linear}

\subsubsection{Random Forest}

\subsubsection{Neural Networks}

\subsection{Simulation}

The machine learning algorithms will be tested on these 12 specifications and their performance evaluated.

\subsection{Real World Data}

Finally, we evaluate machine learning algorithms on real world data. 

\section{Outline of Results Thus Far}

All work has been and will be done in R.

The programs for simulating and tuning the datasets have been written, but not run fully.

The programs for re-sampling the simulated datasets, training models, tuning models and evaluating them has also been written.

If there is enough time and computing resources, Long Term Short Memory models, an extension of neural networks which includes the output of a neural network trained on previous data, will also be evaluated.

\section{Concluding Remarks and Other Considerations}

The computing resources necessary 

\section{Research Timeline}

May, June:
\begin{itemize}
	\item Finalize Simulation of dataset
	\item Begin fitting models to simulated datasets
	\item Organise real world dataset
\end{itemize}

July:
\begin{itemize}
	\item Finish fitting models to simulated datasets
	\item Finish cleaning real world data
	\item Begin fitting models to real world data
\end{itemize}

August:
\begin{itemize}
	\item Collate results from models
	\item Research interpretation of results
	\item First Draft
\end{itemize}

September:
\begin{itemize}
	\item Second Draft
\end{itemize}

October:
\begin{itemize}
	\item Submission
\end{itemize}

\end{document}
