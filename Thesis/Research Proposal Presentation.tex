%%%%%%%%%%%%%%%%%%%%%%%%%%%%%%%%%
%Preamble
%%%%%%%%%%%%%%%%%%%%%%%%%%%%%%%%%
\documentclass{beamer}
\usetheme{Boadilla}

\usepackage[english]{babel}
\usepackage[ruled, vlined]{algorithm2e}

\usepackage{amsfonts}
\usepackage{setspace,graphicx,epstopdf,amsmath}
\usepackage{marginnote, datetime, url, enumitem, subfigure}

%Bibliography Stuff
%Use natbib even though it's old because it's compliant with journal styles
%Actual bibliography style etc are specified where you actually want it
\usepackage{natbib}

%Fluff
\linespread{1.3}

%Neural Network Packages
\usepackage{neuralnetwork}
\usepackage{xpatch}
\makeatletter
% \linklayers have \nn@lastnode instead of \lastnode,
% patch it to replace the former with the latter, and similar for thisnode
\xpatchcmd{\linklayers}{\nn@lastnode}{\lastnode}{}{}
\xpatchcmd{\linklayers}{\nn@thisnode}{\thisnode}{}{}
\makeatother

%Regression Tree
\usepackage{tikz,forest}
\usetikzlibrary{arrows.meta}

\forestset{
	.style={
		for tree={
			base=bottom,
			child anchor=north,
			align=center,
			s sep+=1cm,
			straight edge/.style={
				edge path={\noexpand\path[\forestoption{edge},thick,-{Latex}] 
					(!u.parent anchor) -- (.child anchor);}
			},
			if n children={0}
			{tier=word, draw, thick, rectangle}
			{draw, diamond, thick, aspect=2},
			if n=1{%
				edge path={\noexpand\path[\forestoption{edge},thick,-{Latex}] 
					(!u.parent anchor) -| (.child anchor) node[pos=.2, above] {Y};}
			}{
				edge path={\noexpand\path[\forestoption{edge},thick,-{Latex}] 
					(!u.parent anchor) -| (.child anchor) node[pos=.2, above] {N};}
			}
		}
	}
}

%%TODONOTE commands
\usepackage[colorinlistoftodos]{todonotes}
\newcommand{\smalltodo}[2][] {\todo[caption={#2}, size=\scriptsize,%
	fancyline,#1]{\begin{spacing}{.5}#2\end{spacing}}}
\newcommand{\rhs}[2][]{\smalltodo[color=green!30,#1]{{\bf RS:} #2}}
%%

%Graphs
\usepackage{tikz}
\usepackage{pgfplots}

%Coloured Tables


%%%%%%%%%%%%%%%%%%%%%%%%%%%%%%
%%Title and other fluff, just before document start
%%%%%%%%%%%%%%%%%%%%%%%%%%%%%%

%Hyperref apparently is a big package and causes a lot of issues, so it's recommended to load this last

\usepackage{hyperref}

%Gets rid of the neon green boxes around boxes

\usepackage{xcolor}
\hypersetup{
	colorlinks,
	linkcolor = {red!50!black},
	citecolor = {blue!50!black},
	urlcolor = {blue!80!black}
}

\title{Evaluation of Machine Learning in Finance}
\author{Ze Yu Zhong}


\begin{document}
	
\begin{frame}[plain]
    \maketitle
\end{frame}

\begin{frame}
	\tableofcontents
\end{frame}

%%%%%%%%%%%%%%%%%%%%%%%%%%%%%%%%%%%%%%%%%%%%%%%%%%%%
\section{Problems in Empirical Finance}
%%%%%%%%%%%%%%%%%%%%%%%%%%%%%%%%%%%%%%%%%%%%%%%%%%%%

\begin{frame}
\frametitle{Problems in Empirical Finance}
\begin{itemize}
	\item Regressors can be:
		\begin{itemize}
			\item Non-stationary - information now does not contain information about the future
			\item Persistent - shocks in a series have effects that last for a long time
			\item Cross sectionally correlated - regressors may seem important but are actually the result of a different underlying regressor
			\item Endogeneous - omitted variable bias, etc
		\end{itemize}

\end{itemize}
\end{frame}

\begin{frame}
\frametitle{Problems in Empirical Finance}
\begin{itemize}
	\item Data is not robust - structural breaks are evident in returns data, and many regressors that once performed well do not anymore
	\item Extremely large number of potential factors (regressors) that is still increasing: over 600 documented in the literature
\end{itemize}
\end{frame}

%%%%%%%%%%%%%%%%%%%%%%%%%%%%%%%%%%%%%%%%%%%%%%%%%%%%
\section{What is Machine Learning?}
%%%%%%%%%%%%%%%%%%%%%%%%%%%%%%%%%%%%%%%%%%%%%%%%%%%%

\begin{frame}
\frametitle{What is Machine Learning?}
\begin{itemize}
	\item Statistical/Machine Learning refers to a vast set of tools for understanding data
	\item Building statistical models for predicting outputs based on inputs
	\item Find patterns in datasets
	\item Examples of models: Ordinary Least Squares, LASSO Regression, Generalized Linear Models, Decisions Trees, Neural Networks
\end{itemize}
\end{frame}

%%%%%%%%%%%%%%%%%%%%%%%%%%%%%%%%%%%%%%%%%%%%%%%%%%%%
\section{Why apply Machine Learning in Finance?}
%%%%%%%%%%%%%%%%%%%%%%%%%%%%%%%%%%%%%%%%%%%%%%%%%%%%

\begin{frame}
\frametitle{Why apply Machine Learning in Finance?}
\begin{itemize}
	\item Well suited for prediction
	\item Better equipped to deal with large dimensionality
	\item Capable of capturing non-linear transformations humans cannot realistically find
\end{itemize}
\end{frame}

%%%%%%%%%%%%%%%%%%%%%%%%%%%%%%%%%%%%%%%%%%%%%%%%%%%%
\section{Model Specification}
%%%%%%%%%%%%%%%%%%%%%%%%%%%%%%%%%%%%%%%%%%%%%%%%%%%%

\begin{frame}
\frametitle{Model Overview}
\begin{itemize}
	\item Returns are modelled as an additive error model
	
\end{itemize}
\end{frame}

%%%%%%%%%%%%%%%%%%%%%%%%%%%%%%%%%%%%%%%%%%%%%%%%%%%%
\section{Simulation}
%%%%%%%%%%%%%%%%%%%%%%%%%%%%%%%%%%%%%%%%%%%%%%%%%%%%

\subsection{Real World Observations}
\begin{frame}
\frametitle{Real World Observations}
	content...
\end{frame}

\subsection{Simulation Design}
\begin{frame}
\frametitle{Simulation Design}
content...
\end{frame}

%%%%%%%%%%%%%%%%%%%%%%%%%%%%%%%%%%%%%%%%%%%%%%%%%%%%
\section{Real Data}
%%%%%%%%%%%%%%%%%%%%%%%%%%%%%%%%%%%%%%%%%%%%%%%%%%%%

\begin{frame}
\frametitle{Data Source}
	content...
\end{frame}

%%%%%%%%%%%%%%%%%%%%%%%%%%%%%%%%%%%%%%%%%%%%%%%%%%%%
\section{Results}
%%%%%%%%%%%%%%%%%%%%%%%%%%%%%%%%%%%%%%%%%%%%%%%%%%%%

\begin{frame}
\frametitle{Results}
	content...
\end{frame}

%%%%%%%%%%%%%%%%%%%%%%%%%%%%%%%%%%%%%%%%%%%%%%%%%%%%
\section{Questions and Answers}
%%%%%%%%%%%%%%%%%%%%%%%%%%%%%%%%%%%%%%%%%%%%%%%%%%%%

\begin{frame}
\frametitle{Questions and Answers}
\end{frame}

\end{document}
